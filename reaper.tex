\documentclass[DIV=calc]{scrartcl}
\usepackage[dutch]{babel} %Nederlandse splitsing en titels
\usepackage{mathpazo} %gebruikt palatino, ook in wiskundige formules
\usepackage[T1]{fontenc}
\usepackage[scaled]{beramono} %gebruik beramono voor bvb. computercode
\linespread{1.04} %palatino is iets vetter, dus meer ruimte tussen regels
\usepackage[latin1]{inputenc} %letters met accenten eenvoudig gebruiken        
\usepackage[small,bf,hang]{caption} %mooiere bijschriften bij figuren en tabellen
\usepackage{url} %Webadressen gemakkelijk invoeren
\usepackage{graphicx} %figuren invoegen (pdf, png en jpg)
\usepackage{multirow}
\usepackage{array} %nodig voor mooie vierkanten
 \usepackage{calc}
\usepackage{amsmath}
\usepackage{color}
\usepackage{listings} %nodig voor scilabcode
\definecolor{grijs}{rgb}{0.9,0.9,0.9} %maak lichter grijs door 3 getallen groter te maken 
\usepackage{microtype} %enkele details qua typografie zodat er minder woordsplitsingen nodig zijn
%\usepackage[fixlanguage]{babelbib} %bibliografie aangepast aan taal
%\selectbiblanguage{dutch}
\usepackage{hyperref} %alle verwijzingen en inhoudsopgave klikbaar in PDF
\setkomafont{sectioning}{\rmfamily\bfseries\boldmath} %gewoon palatino in sectietitels enz...
\setkomafont{descriptionlabel}{\rmfamily\bfseries}
%\bibliographystyle{babunsrt}

%%%% onderstaande code is nodig om mooie vierkanten te kunnen maken
%%%% Er wordt een nieuwe environment squarecells gedefinieerd.
\newlength\celldim \newlength\fontheight \newlength\extraheight
\newcounter{sqcolumns}

\newcolumntype{S}{ @{}
  >{\centering \rule[-0.5\extraheight]{0pt}{\fontheight + \extraheight}}
  p{\celldim} @{} }

\newcolumntype{Z}{ @{} >{\centering} p{\celldim} @{} }

\newenvironment{squarecells}[1]
  {\setlength\celldim{2em}%
   \settoheight\fontheight{A}%
   \setlength\extraheight{\celldim - \fontheight}%
   \setcounter{sqcolumns}{#1 - 1}%
   \begin{tabular}{|S|*{\value{sqcolumns}}{Z|}}\hline}
% squarecells tabular goes here
  {\end{tabular}}

\newcommand\nl{\tabularnewline\hline}

\lstset{language=Scilab,basicstyle=\ttfamily \footnotesize,backgroundcolor=\color{grijs},
framerule=0.5pt,tabsize=2}
\lstdefinestyle{inline} {basicstyle=\normalsize \ttfamily}

\frenchspacing



%%%%%%%%%%%%%%%%%% Vanaf hier begin het document en mag je aanpassingen maken %%%%%%%%

\begin{document}
\noindent  \begin{tabular}{ll} 
\multirow{4}{4.5cm}[0mm]{\includegraphics[width=4.5cm]{NieuwLogoGT.pdf} \hfill}
& Maarten Peeters \\
& Ego Vanautgaerden \\
& Yannick Peeters \\
& 3 Ti  \\
& 8 maart 2013 \\
\end{tabular}
\vspace{0.5cm}

\begin{center}
\Large \textsc{Afstudeerproject
Systeem- en Netwerkbeheer: Reaper} 
\end{center}

%\noindent \rule{\textwidth}{0.5pt}
\vspace{0.5cm}
\tableofcontents


\newpage
\section{Idee}
Reaper Digital Audio Workstation(DAW) voorbereiden op gebruik in een klaslokaal.\\
Dit houdt in dat de software op de machines waar de studenten op werken draait en de plugins (de programaatjes die de verwerking van audio doen) te laten draaien op bepaalde servers in een rack.\\
De bedoeling is dat de clients (werkplaatsen studenten) Windows of OSX draaien en de servers Windows of Linux draaien.\\
Hiervoor moeten er scripts geschreven worden die de installatie vergemakkelijken en extra functionaliteit toevoegen (visuele processor verbruik).
\section{Werkverdeling}
\subsection{Ego Vanautgaerden}
\begin{itemize}
		\item Projectleider
		\item Benchmarking
		\item Pxe
		\item Linux Installatiescripts
		\item Python scripting (problemen verhelpen)
		\item How to: Linux
	\end{itemize}
\subsection{Maarten Peeters}
\begin{itemize}
	\item Benchmarking
	\item Python scripting
	\item Psutil
	\item How to: Windows(finale versie)
\end{itemize}
\subsection{Yannick Peeters}
\begin{itemize}
	\item Benchmarking
	\item Windows installatiescript
	\item Python scripting (meewerken aan UI)
	\item How to: Windows (eerste versie)
	\item Verslag in LaTeX
\end{itemize}
\section{Problemen en oplossingen}

\begin{enumerate}
	\item Kan niets uploaden naar de wiki => webontwerp te weinig ruimte voor onze bestanden.
	\begin{itemize}
		\item tijdelijke oplossing, uploaden naar iemand zijn eigen webpagina
		\item whoehoe uploaden bash scripts work
	\end{itemize}
	\item psutil voor de python scripts, installatie was problematisch, heeft 2 uur geduurd op windows en 1 uur op linux
	\begin{itemize}
		\item opgelost
	\end{itemize}
	\item windows batch files zijn vreselijk gelimiteerd in hun kunnen (kan niet automatisch downloaden naar een vaste folder, kan geen zips unzippen (really Bill Gates?)
		\begin{itemize}
		\item unzipper.exe downloaden en gebruiken om de zips te unzippen
	\end{itemize}
	\item problemen met linux wine installaties vanaf command line
	\begin{itemize}
		\item probleem opgelost door de exe's niet direct na installatie te verwijderen en wineconf niet uit te voeren
	\end{itemize}
	\item problemen python ui => is een non threaded stuk waardoor de ui niet wilt updaten
	\begin{itemize}
		\item Gedeeltelijk opgelost. De UI ontvangt de updates, maar de volgorde is willekeurig. Vandaar dat meerdere servers niet gescheiden werden.
	\end{itemize}
	\item instellen van headless server, reamote gebruikt een ui
	\begin{itemize}
		\item x window server instaleren. Deze oplossing was niet ideaal.
	\end{itemize}
	\item Pxe images maken van live cd's
	\begin{itemize}
		\item Fedora heeft een package livecd-tools waar een functie in zit livecd-iso-to-pxe die een compleete pxe map aanmaakt van een gegeven iso file
	\end{itemize}
	\item Damn small linux heeft een verouderde kernel waardoor de onboard netwerkaarten niet werken op de desktops in het school
	\begin{itemize}
		\item drbl uitproberen
	\end{itemize}
	\item Wine installatie zorgt er voor dat live images compleet crashen
	\begin{itemize}
		\item Geen live image gebruiken
	\end{itemize}
	\item MSI installeert de exe's niet, hij plaatst ze gewoon op de schijf
	\begin{itemize}
		\item oude batch files blijven onze eerste installatie wegens falende msi install
	\end{itemize}
	\item wiki staat nog steeeds niet toe om bepaalde dingen up te loaden (ask Geens)
	\begin{itemize}
		\item We hebben de scripts in een zip gestoken en deze zo ge\"upload.
		\item Sinds 06/03 kunnen we ook .bat en .py bestanden uploaden, hierdoor is alles ge\"upload
	\end{itemize}
\end{enumerate}
\section{Documentatie}
\subsection{How to: Linux}
De how to vind je terug op \\
\url{http://wikis.khleuven.be/sysnw/index.php/Reaper#How_To} \\met als naam: "Media:how-to-reaper-lin.pdf"
\subsection{How to: Windows}
De how to vind je terug op \\
\url{http://wikis.khleuven.be/sysnw/index.php/Reaper#How_To} \\met als naam: "Media:how-to-reaper-win.pdf"
\pagebreak
\section{Scripts}
\subsection{Linux installatiescripts}
\subsubsection{Debian - \textit{unstable}}
\begin{lstlisting}[caption={Script-debian-32bit.sh}, label=Debian installatiescript 32bit]
#installatie wget && gcc
apt-get install wget
apt-get install gcc
apt-get install tkinter
 
#downloaden en installeren reaper
winecfg
wget -O /tmp/install.exe http://reaper.fm/files/4.x/reaper432-install.exe
cd /tmp
wget https://wikis.khleuven.be/sysnw/images/8/86/Pref-linux-server.
ReaperConfigZip --no-check-certificate
chmod 666 /tmp/Pref-linux-server.ReaperConfigZip
wine /tmp/install.exe
 
#downloaden en installeren psutil
apt-get install python-dev
wget -O /tmp/psutil.tar.gz 
http://psutil.googlecode.com/files/psutil-0.6.1.tar.gz
cd /tmp
tar -xzf /tmp/psutil.tar.gz
cd psutil-0.6.1
sudo python setup.py install
 
#downloaden VSTplugins
wget -O /tmp/vst.tar.gz www.peetersy.be/reaper/VST_plugins.tar.gz
cd /tmp
tar -xzf /tmp/vst.tar.gz
cp -r /tmp/VST\ plugins/ /home/$SUDO_USER/.wine/drive_c
chmod -R 777 /home/$SUDO_USER/.wine/drive_c/VST\ plugins
 
#iptables instellen om reamote toe te staan
iptables -I INPUT -p tcp --dport 45820 --syn -j ACCEPT
iptables -I INPUT -p udp --dport 45820 -j ACCEPT
 
#downloaden python scripts
cd /tmp
wget https://wikis.khleuven.be/sysnw/images/9/99/Server_scripts.tar.gz 
--no-check-certificate
tar -xzf /tmp/Server_scripts.tar.gz
 
#cleanup
rm /tmp/install.exe
rm /tmp/psutil.tar.gz
rm -r /tmp/psutil-0.6.1
rm /tmp/vst.tar.gz
rm -r /tmp/VST\ plugins
\end{lstlisting}
\begin{lstlisting}[caption={Script-debian-64bit.sh}, label=Debian installatiescript 64bit]
#installatie wget && gcc
apt-get install wget
apt-get install gcc
apt-get install tkinter
 
#downloaden en installeren reaper
winecfg
wget -O /tmp/install.exe 
http://reaper.fm/files/4.x/reaper432_x64-install.exe
cd /tmp
wget https://wikis.khleuven.be/sysnw/images/8/86/Pref-linux-server.
ReaperConfigZip --no-check-certificate
chmod 666 /tmp/Pref-linux-server.ReaperConfigZip
wine /tmp/install.exe
 
#downloaden en installeren psutil
apt-get install python-dev
wget -O /tmp/psutil.tar.gz 
http://psutil.googlecode.com/files/psutil-0.6.1.tar.gz
cd /tmp
tar -xzf /tmp/psutil.tar.gz
cd psutil-0.6.1
sudo python setup.py install
 
#downloaden VSTplugins
wget -O /tmp/vst.tar.gz www.peetersy.be/reaper/VST_plugins.tar.gz
cd /tmp
tar -xzf /tmp/vst.tar.gz
cp -r /tmp/VST\ plugins/ /home/$SUDO_USER/.wine/drive_c
chmod -R 777 /home/$SUDO_USER/.wine/drive_c/VST\ plugins
 
#iptables instellen om reamote toe te staan
iptables -I INPUT -p tcp --dport 45820 --syn -j ACCEPT
iptables -I INPUT -p udp --dport 45820 -j ACCEPT
 
#downloaden python scripts
cd /tmp
wget https://wikis.khleuven.be/sysnw/images/9/99/Server_scripts.tar.gz 
--no-check-certificate
tar -xzf /tmp/Server_scripts.tar.gz
 
#cleanup
rm /tmp/install.exe
rm /tmp/psutil.tar.gz
rm -r /tmp/psutil-0.6.1
rm /tmp/vst.tar.gz
rm -r /tmp/VST\ plugins
\end{lstlisting}




\subsubsection{Fedora}
\begin{lstlisting}[caption={Script-fedora-32bit.sh}, label=Fedora installatiescript 32bit]
#installatie wget && gcc
yum install wget
yum install gcc
yum install tkinter
 
#downloaden en installeren reaper
winecfg
wget -O /tmp/install.exe http://reaper.fm/files/4.x/reaper432-install.exe
cd /tmp
wget https://wikis.khleuven.be/sysnw/images/8/86/Pref-linux-server.
ReaperConfigZip --no-check-certificate
chmod 666 /tmp/Pref-linux-server.ReaperConfigZip
wine /tmp/install.exe
 
#downloaden en installeren psutil
yum install python-devel
wget -O /tmp/psutil.tar.gz 
http://psutil.googlecode.com/files/psutil-0.6.1.tar.gz
cd /tmp
tar -xzf /tmp/psutil.tar.gz
cd psutil-0.6.1
sudo python setup.py install
 
#downloaden VSTplugins
wget -O /tmp/vst.tar.gz www.peetersy.be/reaper/VST_plugins.tar.gz
cd /tmp
tar -xzf /tmp/vst.tar.gz
cp -r /tmp/VST\ plugins/ /home/$SUDO_USER/.wine/drive_c
chmod -R 777 /home/$SUDO_USER/.wine/drive_c/VST\ plugins
 
#iptables instellen om reamote toe te staan
iptables -I INPUT -p tcp --dport 45820 --syn -j ACCEPT
iptables -I INPUT -p udp --dport 45820 -j ACCEPT
service iptables save
 
#downloaden python scripts
cd /tmp
wget https://wikis.khleuven.be/sysnw/images/9/99/Server_scripts.tar.gz 
--no-check-certificate
tar -xzf /tmp/Server_scripts.tar.gz
 
#cleanup
rm /tmp/install.exe
rm /tmp/psutil.tar.gz
rm -r /tmp/psutil-0.6.1
rm /tmp/vst.tar.gz
rm -r /tmp/VST\ plugins
\end{lstlisting}
\begin{lstlisting}[caption={Script-fedora-64bit.sh}, label=Fedora installatiescript 64bit]
#installatie wget && gcc
yum install wget
yum install gcc
yum install tkinter
 
#downloaden en installeren reaper
winecfg
wget -O /tmp/install.exe 
http://reaper.fm/files/4.x/reaper432_x64-install.exe
cd /tmp
wget https://wikis.khleuven.be/sysnw/images/8/86/Pref-linux-server.
ReaperConfigZip --no-check-certificate
chmod 666 /tmp/Pref-linux-server.ReaperConfigZip
wine /tmp/install.exe
 
#downloaden en installeren psutil
yum install python-devel
wget -O /tmp/psutil.tar.gz 
http://psutil.googlecode.com/files/psutil-0.6.1.tar.gz
cd /tmp
tar -xzf /tmp/psutil.tar.gz
cd psutil-0.6.1
sudo python setup.py install
 
#downloaden VSTplugins
wget -O /tmp/vst.tar.gz www.peetersy.be/reaper/VST_plugins.tar.gz
cd /tmp
tar -xzf /tmp/vst.tar.gz
cp -r /tmp/VST\ plugins/ /home/$SUDO_USER/.wine/drive_c
chmod -R 777 /home/$SUDO_USER/.wine/drive_c/VST\ plugins
 
#iptables instellen om reamote toe te staan
iptables -I INPUT -p tcp --dport 45820 --syn -j ACCEPT
iptables -I INPUT -p udp --dport 45820 -j ACCEPT
service iptables save
 
#downloaden python scripts
cd /tmp
wget https://wikis.khleuven.be/sysnw/images/9/99/Server_scripts.tar.gz 
--no-check-certificate
tar -xzf /tmp/Server_scripts.tar.gz
 
#cleanup
rm /tmp/install.exe
rm /tmp/psutil.tar.gz
rm -r /tmp/psutil-0.6.1
rm /tmp/vst.tar.gz
rm -r /tmp/VST\ plugins
\end{lstlisting}
\subsection{Windows installatiescripts}
\begin{lstlisting}[caption={reaper.bat}, label=Windows installatiescript 32bit]
@echo off
echo Bewaar de startende downloads in C:\Temp zonder de namen te wijzigen, 
anders zal dit niet werken!
PAUSE
mkdir C:\Temp
cd C:\Temp
echo Download Python (nodig voor reaper)
start http://www.python.org/ftp/python/2.7.3/python-2.7.3.msi
PAUSE
echo druk op een toets om de Reaper te downloaden.
PAUSE
start http://reaper.fm/files/4.x/reaper432-install.exe
PAUSE
echo druk op een toets om de SWS extensie te downloaden
start http://www.standingwaterstudios.com/reaper/sws_extension.exe
PAUSE
echo druk op een toets om PSUtil te downloaden
start http://psutil.googlecode.com/files/psutil-0.6.1.win32-py2.7.exe
PAUSE
echo unzipper downloaden
start http://stahlforce.com/dev/unzip.exe
PAUSE
echo druk op een toets om de plugins te downloaden
start http://www.peetersy.be/reaper/VST_plugins.zip
PAUSE
echo druk op een toets om de python scripts te downloaden
start http://wikis.khleuven.be/sysnw/images/d/de/Client_scripts.zip
PAUSE
echo druk op een toets om python te installeren
start python-2.7.3.msi
PAUSE
echo druk op een toets om reaper te installeren
start reaper432-install.exe
PAUSE
echo druk op een toets om SWS extensie te installeren
start sws_extension.exe
PAUSE
echo druk op een toets om PSUtil te installeren
start psutil-0.6.1.win32-py2.7.exe
PAUSE
echo druk op een toets om de plugins te installeren
unzip.exe -d C:\Temp\plugins VST_plugins
unzip.exe -d C:\Temp\client_scripts Client_scripts
mkdir "%userprofile%\Desktop\Python Scripts Reaper"
XCOPY "C:\Temp\client_scripts" 
"%userprofile%\Desktop\Python Scripts Reaper" /S /Y
mkdir C:\VST_Plugins
XCOPY "C:\Temp\plugins\VST plugins" C:\VST_plugins /S /Y
echo even de tijdelijke bestanden opkuisen, 
gelieve dus J te antwoorden op de volgende vraag.
cd C:\
RD C:\Temp\ /S
\end{lstlisting}
\begin{lstlisting}[caption={reaper64.bat}, label=Windows installatiescript 64bit]
@echo off
echo Bewaar de startende downloads in C:\Temp zonder de namen te wijzigen, 
anders zal dit niet werken!
PAUSE
mkdir C:\Temp
cd C:\Temp
echo Download Python (nodig voor reaper)
start http://www.python.org/ftp/python/2.7.3/python-2.7.3.amd64.msi
PAUSE
echo druk op een toets om de Reaper te downloaden.
PAUSE
start http://www.reaper.fm/files/4.x/reaper432_x64-install.exe
PAUSE
echo druk op een toets om de SWS extensie te downloaden
start http://www.standingwaterstudios.com/reaper/sws_extension_x64.exe
PAUSE
echo druk op een toets om PSUtil te downloaden
start http://psutil.googlecode.com/files/psutil-0.6.1.win-amd64-py2.7.exe
PAUSE
echo unzipper downloaden
start http://stahlforce.com/dev/unzip.exe
PAUSE
echo druk op een toets om de plugins te downloaden
start http://www.peetersy.be/reaper/VST_plugins.zip
PAUSE
echo druk op een toets om de python scripts te downloaden
start http://wikis.khleuven.be/sysnw/images/d/de/Client_scripts.zip
PAUSE
echo druk op een toets om python te installeren
start python-2.7.3.amd64.msi
PAUSE
echo druk op een toets om reaper te installeren
start reaper432_x64-install.exe
PAUSE
echo druk op een toets om SWS extensie te installeren
start sws_extension_x64.exe
PAUSE
echo druk op een toets om PSUtil te installeren
start psutil-0.6.1.win-amd64-py2.7.exe
PAUSE
echo druk op een toets om de plugins te installeren
unzip.exe -d C:\Temp\plugins VST_plugins
unzip.exe -d C:\Temp\client_scripts Client_scripts
mkdir "%userprofile%\Desktop\Python Scripts Reaper"
XCOPY "C:\Temp\client_scripts" 
"%userprofile%\Desktop\Python Scripts Reaper" /S /Y
mkdir C:\VST_Plugins
XCOPY "C:\Temp\plugins\VST plugins" C:\VST_plugins /S /Y
echo even de tijdelijke bestanden opkuisen, 
gelieve dus J te antwoorden op de volgende vraag.
cd C:\
RD C:\Temp\ /S
\end{lstlisting}

\subsection{Python scripts}
\begin{lstlisting}[caption={cputest.py}, label=Processor- en geheugengebruik opvragen van pc]
import psutil
class Usage:
    def __init__(self): pass
 
    def ask_cpu(self):
        return str(psutil.cpu_percent(interval=0.1, percpu = False))
 
    def ask_cpu_spec(self, process):
        pname = process
        for p in psutil.process_iter():
            if p.name == pname:
                return str(round(p.get_cpu_percent(interval = 0.1,),2))
 
    def ask_ram(self):
        return str(psutil.virtual_memory().percent)
 
    def ask_ram_spec(self, process):
        pname = process
        for p in psutil.process_iter():
            if p.name == pname:
                return str(round(p.get_memory_percent(),1))
 
    def ask_num_proc(self):
        return psutil.NUM_CPUS
\end{lstlisting}
\begin{lstlisting}[caption={broadcast\_send.py}, label=Server laten broadcasten op bepaalde poort]
MYPORT = 50000
 
import sys, time
from socket import *
 
class Caster:
    s = socket(AF_INET, SOCK_DGRAM)
    s.bind(('', 0))
    s.setsockopt(SOL_SOCKET, SO_BROADCAST, 1)
 
    def send(self,text):
        data = repr(text)
        self.s.sendto(data, ('<broadcast>', MYPORT))
\end{lstlisting}
\begin{lstlisting}[caption={launcher.py}, label=launcher langs serverzijde]
class Launcher:
    from broadcast_send import Caster
    from cputest import Usage
    import time
 
    use = Usage()
    bsen = Caster()
 
    while(1):
        try:
            bsen.send("cpu-reaper: " + use.ask_cpu_spec("reamote.exe"))
            bsen.send("ram-reaper: " + use.ask_ram_spec("reamote.exe"))  
            bsen.send("cpu: " + use.ask_cpu())
            bsen.send("ram: " + use.ask_ram())
            bsen.send("cores: " + str(use.ask_num_proc()))
            time.sleep(1)
        except(TypeError):
            time.sleep(1)
            print "Staat reamote wel op?"
\end{lstlisting}
\begin{lstlisting}[caption={launcher\_rec.py}, label=launcher langs clientzijde]
from buffer import Buff
 
b = Buff()
b.voerUit()
\end{lstlisting}
\begin{lstlisting}[caption={buffer.py}, label=Threads laten starten van een functie van de Receiver en een functie van de MainFrame]
from broadcast_receive import Receiver
from pythonUI import MainFrame
import thread
 
class Buff:
    a = Receiver()
    m = MainFrame()
    m.addrec(a)
 
    def __init__(self):
        thread.start_new_thread(self.a.run,())
        thread.start_new_thread(self.m.werk,())
 
    def voerUit(self):
        t1 = thread(target = self.a.run())
        t2 = thread(target = self.m.werk())
        t1.start()
        t2.start()
        t1.join()
        t2.join()
\end{lstlisting}
\begin{lstlisting}[caption={pyhtonUI.py}, label=UI van het processor- en geheugengebruik van de server]
from Tkinter import *
 
class MainFrame:    
    bgcolor = '#321000'
    fgcolor = '#000fff000'
    f = 12
    m = 0
    h = 0
    rcv = 0
    root = Tk()
    root.geometry('290x150+120+120')
    root.wm_title('Processor and Memory Usage')
    root.configure(background= bgcolor)
    cpugrid = StringVar()
    ramgrid = StringVar()
    cpureapergrid = StringVar()
    ramreapergrid = StringVar()
    coresgrid = StringVar()
    runtimegrid = StringVar()
 
    label1 = Label(root,textvariable=cpugrid,bg= bgcolor,
		fg= fgcolor,font=("Helvetica",f))
    label2 = Label(root,textvariable=ramgrid,bg= bgcolor,
		fg= fgcolor,font=("Helvetica",f))
    label3 = Label(root,textvariable=cpureapergrid,bg= bgcolor,
		fg= fgcolor,font=("Helvetica",f))
    label4 = Label(root,textvariable=ramreapergrid,bg= bgcolor,
		fg= fgcolor,font=("Helvetica",f))
    label5 = Label(root,textvariable=coresgrid,bg= bgcolor,
		fg= fgcolor,font=("Helvetica",f))
    label6 = Label(root,textvariable=runtimegrid,bg= bgcolor,
		fg= fgcolor,font=("Helvetica",f))
 
    label1.grid(sticky=W)
    label2.grid(sticky=W)
    label3.grid(sticky=W)
    label4.grid(sticky=W)
    label5.grid(sticky=W)
    label6.grid(sticky=W)
 
 
    def __init__(self):
        self.cpugrid.set("CPU Usage:" + " " + str(0) + "%")
        self.ramgrid.set("RAM Usage:" + " " + str(0) + "%")
        self.cpureapergrid.set("CPU-REAPER Usage:" + " " + str(0) + "%")
        self.ramreapergrid.set("RAM-REAPER Usage:" + " " + str(0) + "%")
        self.coresgrid.set("CORES:" + " " + str(0) + "%")
 
    def addrec(self, rec):
        self.rcv = rec
 
def DoeIets(self,n, master):       
        self.root.wm_title(self.rcv.getAddr())
        self.cpugrid.set("CPU Usage:" + " " + self.rcv.getCpu() + "%")
        self.ramgrid.set("RAM Usage:" + " " + self.rcv.getRam() + "%")
        self.cpureapergrid.set("CPU-REAPER Usage: " 
				+ self.rcv.getCpuReaper() + "%")
        self.ramreapergrid.set("RAM-REAPER Usage: " 
				+ self.rcv.getRamReaper() + "%")
        self.coresgrid.set("CORES:" + " " + self.rcv.getCores())
 
        if(self.h>0):
            self.runtimegrid.set(str(self.h) + " hours " 
						+ str(self.m) + " minutes " + str(n) + " seconds running")
 
        elif(self.m>0):
            self.runtimegrid.set(str(self.m) + " minutes " 
						+ str(n) + " seconds running")
 
        else:
            self.runtimegrid.set(str(n) + " seconds running")
 
        n += 1
        if (n == 60):
            n -= 60
            self.m += 1
 
        if (self.m == 60):
            self.m -= 60
            self.h += 1
 
        if n >=0 :
            self.root.after(1000, self.DoeIets, n, master)
 
    def werk(self):       
        self.DoeIets(0, self.root)
        self.root.mainloop()
\end{lstlisting}
\begin{lstlisting}[caption={broadcast\_recieve.py}, label=luister op poort en ontvangt data]
import socket, traceback
  
class Receiver:
 
    adresses = []
    cpu = "0"
    ram = "0"
    cpureaper = "0"
    ramreaper = "0"
    cores = "0"
    addr = "Processor and Memory Usage"
 
    def __init__(self):
        host = ''                               # Bind to all interfaces
        port = 50000
 
        self.s = socket.socket(socket.AF_INET, socket.SOCK_DGRAM)
        self.s.setsockopt(socket.SOL_SOCKET, socket.SO_REUSEADDR, 1)
        self.s.setsockopt(socket.SOL_SOCKET, socket.SO_BROADCAST, 1)
        self.s.bind((host, port))
 
 
    def getword (self,word,text):
        if (word + ": " in text):
            a = text.split(':')[1]
            return a.split("'")[0]
  
    def run(self):
        while 1:
            try:
                message, address = self.s.recvfrom(50000)
                try:
                    self.adresses.index(address)
                    self.setAddr(address)                        
                except (ValueError):
                    self.adresses.append(address)
 
                for item in message.split("\n"):
                    if (self.getword("cpu", item) != None):
                        self.cpu = self.getword("cpu", item)
 
                for item in message.split("\n"):
                    if (self.getword("ram", item) != None):
                        self.ram = self.getword("ram", item)
 
                for item in message.split("\n"):
                    if (self.getword("cpu-reaper", item) != None):
                        self.cpureaper = self.getword("cpu-reaper", item)
 
                for item in message.split("\n"):
                    if (self.getword("ram-reaper", item) != None):
                        self.ramreaper = self.getword("ram-reaper", item)
 
                for item in message.split("\n"):
                    if (self.getword("cores", item) != None):
                        self.cores = self.getword("cores", item)                                        
 
            except (KeyboardInterrupt, SystemExit):
                raise
            except:
                traceback.print_exc()
 
    def setAddr(self, naddr):
        self.addr = naddr
 
    def getAddr(self):
        return self.addr
 
    def getCpu(self):
        return self.cpu
 
    def getRam(self):
        return self.ram
 
    def getCpuReaper(self):
        return self.cpureaper
 
    def getRamReaper(self):
        return self.ramreaper
 
    def getCores(self):
        return self.cores
\end{lstlisting}
\section{Bronnen}
\subsection{Websites}
\begin{itemize}
	\item \url{www.reaper.fm}
	\item \url{www.forum.cockos.com}
	\item \url{www.standingwaterstudios.com}
	\item \url{www.reaperblog.net}
	\item \url{www.homerecordingshow.com}
	\item \url{www.cisco.com}
	\item \url{www.fedoraproject.org}
	\item \url{www.ubuntu.com}
	\item \url{www.damnsmalllinux.org}
	\item \url{www.partedmagic.com}
	\item \url{www.stackoverflow.com}
	\item \url{www.debian.org}
	\item \url{www.debianhelp.co.uk}
	\item \url{www.wikipedia.org}
	\item \url{www.kegel.com}
	\item \url{www.linux-sxs.org}
	\item \url{www.ubuntu-tutorials.com}
	\item \url{www.stdout.org/~winston/latex}
	\item \url{www.howtoforge.com}
	\item \url{www.cyberciti.biz}
	\item \url{www.digitalherpes.wordpress.com}
	\item \url{www.symantec.com}
	\item \url{github.com/certik/python-2.7/blob/master/Demo/sockets}
	\item \url{code.google.com/p/psutil}
	\item \url{msdn.microsoft.com/en-us/library/windows/desktop}
	\item \url{forums.techguy.org}
	\item \url{www.askubuntu.com}
	\item \url{www.ubuntuforums.org}
	\item \url{www.stackoverflow.com}
	\item \url{www.effbot.org}
	\item \url{www.academicdownload.com}
	\item \url{www.daniweb.com/software-development/python}
	\item \url{www.python-course.eu}
	\item \url{www.pydev.org}
	\item \url{www.tutorialspoint.com/python}
	\item \url{www.wellho.net}
	\item \url{www.saltycrane.com}
	\item \url{www.serverfault.com}
	\item \url{www.advancedinstaller.com}
	\item \url{www.emcosoftware.com}
	\item \url{www.dennisbareis.com}
	\item \url{wix.tramontana.co.hu}
	\item \url{wix.sourceforge.net}
	\item \url{wix.codeplex.com}
	\item \url{technet.microsoft.com}
	\item \url{www.superuser.com}
	\item \url{discussion.dreamhost.com}
	\item \url{http://www.ss64.com}
	\item \url{http://curl.haxx.se}
	\item \url{http://docs.python.org}
	\item \url{http://uk.answers.yahoo.com}
	\item \url{http://wiki.python.org/moin/TkInter}
\end{itemize}
\subsection{Personen}
\begin{itemize}
	\item Jasper Piens
	\item Koen Wilms
	\item Pieter Geens
	\item Hans Maes
	\item Gerben Dierick
\end{itemize}
\end{document}